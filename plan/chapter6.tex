\documentclass{TIJMUjiaoanLL}
\pagestyle{empty}

\begin{document}

\kecheng{分子生物计算}
\neirong{子程序和Bugs \ / 第6章}
\jiaoshi{伊现富}
\zhicheng{讲师}
\riqi{2019年9月25\&27日10:00-11:40\&13:30-15:10}
\duixiang{生物医学工程与技术学院2017级生信班(本)}
\renshu{28}
\fangshi{理论讲授}
\xueshi{4}
\jiaocai{Perl语言在生物信息学中的应用——基础篇}

\firstHeader
\maketitle
\thispagestyle{empty}

\mudi{
\begin{itemize}
  \item 掌握:子程序的使用——定义、调用、数据传递等。
  \item 熟悉:Perl调试器的基本用法。
  \item 了解:模块的编写与使用。
  \item 自学:Perl调试器的高级用法。
\end{itemize}
}

\fenpei{
\begin{itemize}
  \item (5')引言与导入:对子程序和Perl调试器进行简介。
  \item (30')子程序:讲解子程序的基本用法,包括定义、调用、参数、作用域等。
  \item (10')命令行参数和数组:讲解命令行参数的使用,以及相关的特殊变量。
  \item (45')传递数据给子程序:讲解传递数据给子程序的方法,对值传递和引用传递进行比较。
  \item (10')模块和子程序库:介绍模块的编写和使用。
  \item (75')修复Bugs:总结调试程序的方法,讲解Perl调试器的使用。
  \item (5')总结与答疑:总结授课内容中的知识点与技能,解答学生疑问。
\end{itemize}
}

\zhongdian{
\begin{itemize}
  \item 重点:子程序的基本用法;传递数据给子程序的方法。
  \item 难点:传递数据给子程序的方法;Perl调试器的使用。
  \item 解决策略:通过实例演示帮助学生理解、记忆。
\end{itemize}
}

\waiyu{
\vspace*{-10pt}
\begin{multicols}{2}
子程序(subroutine)

调试器(debugger)

程序错误(bug)

参数(argument/parameter)

值(value)

引用(reference)

模块(module)

库(library)
\end{multicols}
\vspace*{-10pt}
}

\fuzhu{
\begin{itemize}
  \item 多媒体:子程序的基本用法;模块的使用。
  \item 板书:传递数据给子程序的方法。
  \item 演示:Perl调试器的使用。
\end{itemize}
}

\sikao{
\vspace*{-10pt}
\begin{multicols}{2}
\begin{itemize}
  \item 如何定义和调用子程序?
  \item 举例说明作用域的概念。
  \item 如何获取命令行参数?
  \item 举例说明传递数据给子程序的方法。
  \item 总结调试Perl程序的方法。
  \item 如何使用Perl调试器调试程序?
\end{itemize}
\end{multicols}
\vspace*{-10pt}
}

\cankao{
\begin{itemize}
  \item Beginning Perl for Bioinformatics, James Tisdall, O'Reilly Media, 2001.
  \item Perl语言入门(第六版),Randal L. Schwartz, brian d foy \& Tom Phoenix著,盛春\ 译,东南大学出版社,2012。
  \item Mastering Perl for Bioinformatics, James Tisdall, O'Reilly Media, 2003.
  \item 维基百科等网络资源。
\end{itemize}
}

\firstTail

\newpage
\otherHeader

\begin{enumerate}
  \item 引言与导入(5分钟)
    \begin{itemize}
      \item 子程序:对程序进行结构化组织
      \item Perl调试器:使用“慢镜头”检查程序的行为
    \end{itemize}
  \item \textcolor{red}{【重点】}子程序(30分钟)
    \begin{enumerate}
      \item 简介
	\begin{itemize}
	  \item 简介:程序中的程序 $\Longrightarrow$ 包裹代码、起个名字、传递数据、运算返回结果
	  \item 优势:一次编写、多次使用 $\Longrightarrow$ 简短易理解、稳健易测试、灵活模块化
	  \item 原则:只做一件事情并把它做好 $\Longrightarrow$ 通用且有用、不止使用一次、代码不超过一页
	\end{itemize}
      \item 编写\textcolor{red}{(通过实例演示讲解)}
	\begin{itemize}
	  \item 程序分块:主程序(从开头到exit命令)+子程序(其余部分)
	  \item 定义与调用:集中定义,通过子程序名(和参数)进行调用
	\end{itemize}
\vspace*{-1em}
\begin{multicols}{2}
      \item 定义
	\begin{itemize}
	  \item 语法:sub +名字+代码块
	  \item 两类变量:传递给子程序的参数,子程序中声明的变量
	  \item 返回值:return,标量、列表、数组等
	\end{itemize}
\begin{verbatim}
sub addACGT {
  my ($dna) = @_;
  $dna .= 'ACGT';
  return $dna;
}
\end{verbatim}
\end{multicols}
\vspace*{-1em}
      \item 参数
	\begin{itemize}
	  \item 传递:变量名无关紧要,关键是值及其顺序
	  \item 收集:特殊变量 \verb|@_| 数组
	\end{itemize}
      \item 作用域\textcolor{red}{(比较使用my前后程序的输出)}
	\begin{itemize}
	  \item 作用域:把变量隐藏起来,使它们仅局限在程序的特定部分
	  \item 词法作用域:my,把变量限制在使用它们的代码块中
	\end{itemize}
    \end{enumerate}
  \item 命令行参数和数组(10分钟)
\vspace*{-1em}
\begin{multicols}{2}
    \begin{enumerate}
      \item 命令行参数
	\begin{itemize}
	  \item 非交互、自动化
	  \item \verb|$0|:程序名
	  \item \verb|@ARGV|:所有的命令行参数
	\end{itemize}
      \item 数组使用
	\begin{itemize}
	  \item 起始索引为0
	  \item 使用 \verb|$|而非 \verb|@|提取元素
	  \item 在中括号中放置下标
	\end{itemize}
    \end{enumerate}
\end{multicols}
\vspace*{-1em}
  \item \textcolor{red}{【重点、难点】}传递数据给子程序(45分钟)\textcolor{red}{(通过实例进行演示,并对两者进行比较)}
    \begin{enumerate}
      \item 通过值传递
	\begin{itemize}
	  \item 实例:\verb|simple_sub($i);|
	  \item 说明
	    \begin{itemize}
	      \item 调用子程序时参数的值被复制并传递给子程序
	      \item 子程序中值的变化不会影响到主程序中相应参数的值
	    \end{itemize}
	  \item 适用:单个标量、标量列表、单个数组等
	\end{itemize}
      \item 通过引用传递
	\begin{itemize}
	  \item 实例:\verb|reference_sub(\@i, \@j);|(\textcolor{red}{vs.} \verb|reference_sub(@i, @j);|
	  \item 说明
	    \begin{itemize}
	      \item 引用:在变量名前加 \verb|\|,存储在标量变量中的一种特殊类型的数据
	      \item 收集:从 \verb|@_|中读取参数时要保存为标量变量
	      \item 使用:先解引用后使用
	      \item 解引用:在引用前添加表明变量类型的符号(\verb|$|、\verb|@|、\verb|%|)
	      \item 注意:子程序中对参数变量值的操作会影响到主程序中参数的值
	    \end{itemize}
	  \item 适用:复杂参数(混合标量、数组和散列)
	\end{itemize}
    \end{enumerate}

\otherTail
\newpage
\otherHeader

  \item 模块和子程序库(10分钟)
    \begin{enumerate}
      \item 模块
        \begin{itemize}
           \item 作用:避免繁琐、重复地复制粘贴子程序
           \item 收集:把常用的子程序统一放在一个文件中
	   \item 模块:最后一行必须是 \verb|1;|,后缀为 \verb|.pm|
	   \item 使用:在程序顶部使用use语句,不需要后缀
	   \item 实例:\verb|use BeginPerlBioinfo;|
        \end{itemize}
      \item 子程序库
	\begin{itemize}
	  \item 作用:指定保存模块的绝对路径
	  \item 实例:\verb|use lib '/home';|
	\end{itemize}
    \end{enumerate}
  \item 修复Bugs(75分钟)
    \begin{enumerate}
      \item bug
	\begin{itemize}
	  \item 简介:程序设计术语,程序错误/漏洞,源于历史中的典故
	  \item 常见:括号没配对、没用分号结尾、拼写错误、索引错误、……
	\end{itemize}
      \item \verb|use warnings;|和\verb|use strict;|
	\begin{itemize}
	  \item \verb|use warnings;|:开启警告功能
	  \item \verb|use strict;|:强制使用my声明变量……
	\end{itemize}
      \item 使用注释和print语句
	\begin{itemize}
	  \item 选择性注释:适用于没有精确定位错误、但是直到大体范围时
	  \item 添加print语句:打印出变量的值,适用于差不多直到错误之处时
	\end{itemize}
      \item \textcolor{red}{【难点】}Perl调试器\textcolor{red}{(通过实例演示其使用)}
	\begin{itemize}
	  \item 启动与停止:perl -d script.pl;输入q退出
	  \item 使用帮助:man perldebug;h;h h;h CMD
	  \item 常用命令:p、n、s、v、b、c、R、……
	\end{itemize}
    \end{enumerate}
  \item 总结与答疑(5分钟)
    \begin{enumerate}
      \item 知识点
	\begin{itemize}
	  \item 子程序:定义,调用,返回值,参数,作用域
	  \item 命令行参数:特殊变量,提取数组元素
	  \item 传递数据给子程序:通过值 vs. 通过引用
	  \item 模块:编写、使用,指定库目录
	  \item 调试:\verb|use warnings;|和\verb|use strict;|,注释和print,调试器
	  \item Perl调试器:启动和停止,常用命令
	\end{itemize}
      \item 技能
	\begin{itemize}
	  \item 能够熟练使用子程序
	  \item 能够调试Perl程序
	  \item 能够熟练使用Perl调试器
	\end{itemize}
    \end{enumerate}
\end{enumerate}

\otherTail

%\parpic[fr]{\includegraphics[width=\textwidth]{}}

\end{document}
