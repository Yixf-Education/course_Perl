\documentclass{TIJMUjiaoanSY}
\pagestyle{empty}

\begin{document}

\kecheng{分子生物计算}
\shiyan{实验4 \ 基序和循环}
\jiaoshi{伊现富}
\zhicheng{讲师}
\riqi{2016年12月1日10:00-12:00}
\duixiang{生物医学工程与技术学院2013级生信班(本)}
\renshu{30}
\leixing{验证型}
\fenzu{一人一机}
\xueshi{2}
\jiaocai{Perl语言在生物信息学中的应用——基础篇}

\firstHeader
\maketitle
\thispagestyle{empty}

\mudi{
\begin{itemize}
  \item 了解Perl语言中的流程控制。
  \item 熟悉Perl语言中的if-elsif-else和while;写入文件的方法。
  \item 掌握Perl语言在生物序列数据处理中的应用。
\end{itemize}
}

\fenpei{
\begin{itemize}
  \item (5')流程控制:总结Perl语言中的条件判断和循环语句。
  \item (5')字符串操作:总结Perl语言中常见的字符串操作。
  \item (5')写入文件:总结在Perl语言中写入文件的基本步骤。
  \item (85')实验操作:应用Perl语言处理生物序列数据。
\end{itemize}
}

\cailiao{
\begin{itemize}
  \item 主要仪器:一台安装有Perl语言(Linux操作系统)的计算机。
\end{itemize}
}
\zhongdian{
\begin{itemize}
  \item 重点难点:应用Perl语言处理生物序列数据。
  \item 解决策略:通过演示进行学习,通过练习熟练掌握。
\end{itemize}
}

\sikao{
\begin{itemize}
  \item 总结Perl语言中的流程控制。
  \item 总结Perl语言中常见的字符串操作。
  \item 总结Perl语言中写入文件的基本步骤。
\end{itemize}
}

\cankao{
\begin{itemize}
  \item Beginning Perl for Bioinformatics, James Tisdall, O'Reilly Media, 2001.
  \item Perl语言入门(第六版),Randal L. Schwartz, brian d foy \& Tom Phoenix著,盛春\ 译,东南大学出版社,2012。
  \item Mastering Perl for Bioinformatics, James Tisdall, O'Reilly Media, 2003.
  \item 维基百科等网络资源。
\end{itemize}
}

\firstTail

\newpage
\otherHeader

\begin{enumerate}
\vspace{-1em}
\begin{multicols}{2}
  \item 流程控制(5分钟)
    \begin{enumerate}
      \item 条件判断:if,if-else,unless
      \item 循环:while,for,foreach
    \end{enumerate}
  \item 字符串操作(5分钟)
    \begin{enumerate}
      \item 数组变标量:join
      \item 字符串变数组:split
      \item 提取子字符串:substr
      \item 计数:tr
    \end{enumerate}
  \item 写入文件(5分钟)
    \begin{enumerate}
      \item 关联文件和文件句柄
      \item 通过文件句柄写入数据
      \item 解关联文件和文件句柄
    \end{enumerate}
\end{multicols}
\vspace{-1em}
  \item 实验操作(85分钟)
    \begin{enumerate}
      \item if-elsif-else
\begin{verbatim}
#!/usr/bin/perl -w

$word = 'MNIDDKL';
if ( $word eq 'QSTVSGE' ) {
    print "QSTVSGE\n";
}
elsif ( $word eq 'MRQQDMISHDEL' ) {
    print "MRQQDMISHDEL\n";
}
elsif ( $word eq 'MNIDDKL' ) {
    print "MNIDDKL-the magic word!\n";
}
else {
    print "Is \"$word\" a peptide? This program is not sure.\n";
}
\end{verbatim}
      %\item 使用while从文件中读取蛋白质序列数据
%\begin{verbatim}
%#!/usr/bin/perl -w

%unless ( open( PROTEINFILE, $proteinfilename ) ) {
    %print "Could not open file $proteinfilename!\n";
    %exit;
%}
%while ( $protein = <PROTEINFILE> ) {
    %print "  ######  Here is the next line of the file:\n";
    %print $protein;
%}
%close PROTEINFILE;
%\end{verbatim}
      \item 在蛋白质序列中查找用户指定的基序
\begin{verbatim}
#!/usr/bin/perl -w

$proteinfilename = <STDIN>;
chomp $proteinfilename;
unless ( open( PROTEINFILE, $proteinfilename ) ) {
    print "Cannot open file \"$proteinfilename\"\n\n";
    exit;
}
@protein = <PROTEINFILE>;
close PROTEINFILE;

$protein = join( '', @protein );
$protein =~ s/\s//g;
do {
    print "Enter a motif to search for: ";
    $motif = <STDIN>;
    chomp $motif;
    if ( $protein =~ /$motif/ ) {
        print "I found it!\n\n";
    }
    else {
        print "I couldn\'t find it.\n\n";
    }
} until ( $motif =~ /^\s*$/ );
\end{verbatim}

\otherTail
\newpage
\otherHeader

      \item 对DNA序列中的碱基进行计数(策略一)
\begin{verbatim}
#!/usr/bin/perl -w

print "Please type the filename of the DNA sequence data: ";
$dna_filename = <STDIN>;
chomp $dna_filename;
unless ( open( DNAFILE, $dna_filename ) ) {
    print "Cannot open file \"$dna_filename\"\n\n";
    exit;
}
@DNA = <DNAFILE>;
close DNAFILE;

$DNA = join( '', @DNA );
$DNA =~ s/\s//g;
@DNA = split( '', $DNA );

$count_of_A = 0;
$count_of_C = 0;
$count_of_G = 0;
$count_of_T = 0;
$errors     = 0;

foreach $base (@DNA) {
    if ( $base eq 'A' ) {
        ++$count_of_A;
    }
    elsif ( $base eq 'C' ) {
        ++$count_of_C;
    }
    elsif ( $base eq 'G' ) {
        ++$count_of_G;
    }
    elsif ( $base eq 'T' ) {
        ++$count_of_T;
    }
    else {
        print "!!!!!!!! Error - I don\'t recognize this base: $base\n";
        ++$errors;
    }
}

print "A = $count_of_A\n";
print "C = $count_of_C\n";
print "G = $count_of_G\n";
print "T = $count_of_T\n";
print "errors = $errors\n";
\end{verbatim}

\otherTail
\newpage
\otherHeader

      \item 对DNA序列中的碱基进行计数(策略二)
\begin{verbatim}
#!/usr/bin/perl -w

print "Please type the filename of the DNA sequence data: ";
$dna_filename = <STDIN>;
chomp $dna_filename;
unless ( -e $dna_filename ) {
    print "File \"$dna_filename\" doesn\'t seem to exist!!\n";
    exit;
}
unless ( open( DNAFILE, $dna_filename ) ) {
    print "Cannot open file \"$dna_filename\"\n\n";
    exit;
}
@DNA = <DNAFILE>;
close DNAFILE;

$DNA = join( '', @DNA );
$DNA =~ s/\s//g;
$count_of_A = 0;
$count_of_C = 0;
$count_of_G = 0;
$count_of_T = 0;
$errors     = 0;

for ( $position = 0 ; $position < length $DNA ; ++$position ) {
    $base = substr( $DNA, $position, 1 );
    if ( $base eq 'A' ) {
        ++$count_of_A;
    }
    elsif ( $base eq 'C' ) {
        ++$count_of_C;
    }
    elsif ( $base eq 'G' ) {
        ++$count_of_G;
    }
    elsif ( $base eq 'T' ) {
        ++$count_of_T;
    }
    else {
        print "Error - I don\'t recognize this base: $base\n";
        ++$errors;
    }
}
print "A = $count_of_A\n";
print "C = $count_of_C\n";
print "G = $count_of_G\n";
print "T = $count_of_T\n";
print "errors = $errors\n";
\end{verbatim}

\otherTail
\newpage
\otherHeader

      \item 对DNA序列中的碱基进行计数(策略三),结果保存到文件
\begin{verbatim}
#!/usr/bin/perl -w

print "Please type the filename of the DNA sequence data: ";
$dna_filename = <STDIN>;
chomp $dna_filename;
unless ( -e $dna_filename ) {
    print "File \"$dna_filename\" doesn\'t seem to exist!!\n";
    exit;
}
unless ( open( DNAFILE, $dna_filename ) ) {
    print "Cannot open file \"$dna_filename\"\n\n";
    exit;
}
@DNA = <DNAFILE>;
close DNAFILE;

$DNA = join( '', @DNA );
$DNA =~ s/\s//g;

$a = 0; $c = 0; $g = 0; $t = 0; $e = 0;

while ( $DNA =~ /a/ig )       { $a++ }
while ( $DNA =~ /c/ig )       { $c++ }
while ( $DNA =~ /g/ig )       { $g++ }
while ( $DNA =~ /t/ig )       { $t++ }
while ( $DNA =~ /[^acgt]/ig ) { $e++ }
print "A=$a C=$c G=$g T=$t errors=$e\n";

$outputfile = "countbase";
unless ( open( COUNTBASE, ">$outputfile" ) ) {
    print "Cannot open file \"$outputfile\" to write to!!\n\n";
    exit;
}
print COUNTBASE "A=$a C=$c G=$g T=$t errors=$e\n";
close(COUNTBASE);
\end{verbatim}
      \item Perl对数字和字符串的智能化处理
\begin{verbatim}
#!/usr/bin/perl -w

$num = 1234;
$str = '1234';
print $num, " ", $str, "\n";

$num_or_str = $num + $str;
print $num_or_str, "\n";

$num_or_str = $num . $str;
print $num_or_str, "\n";
\end{verbatim}
    \end{enumerate}
\end{enumerate}

\otherTail


\end{document}
