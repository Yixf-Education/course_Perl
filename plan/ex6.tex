\documentclass{TIJMUjiaoanSY}
\pagestyle{empty}

\begin{document}

\kecheng{分子生物计算}
\shiyan{实验6 \ 模拟DNA突变}
\jiaoshi{伊现富}
\zhicheng{讲师}
\riqi{2019年10月9日13:30-15:10}
\duixiang{生物医学工程与技术学院2017级生信班(本)}
\renshu{28}
\leixing{验证型}
\fenzu{一人一机}
\xueshi{2}
\jiaocai{Perl语言在生物信息学中的应用——基础篇}

\firstHeader
\maketitle
\thispagestyle{empty}

\mudi{
\begin{itemize}
  \item 了解DNA突变中的点突变。
  \item 熟悉Perl语言中的随机数生成器。
  \item 掌握随机选取数组元素和随机选取字符串位置的方法。
\end{itemize}
}

\fenpei{
\begin{itemize}
  \item (10')随机:回顾Perl语言中随机的相关知识点。
  \item (80')实验操作:编写Perl程序模拟DNA突变。
\end{itemize}
}

\cailiao{
\begin{itemize}
  \item 主要仪器:一台安装有Perl语言(Linux操作系统)的计算机。
\end{itemize}
}
\zhongdian{
\begin{itemize}
  \item 重点难点:随机选取数组元素;随机选取字符串位置。
  \item 解决策略:通过演示进行学习,通过练习熟练掌握。
\end{itemize}
}

\sikao{
\begin{itemize}
  \item 在Perl语言中如何设置随机数种子?
  \item 如何随机选取数组元素?
  \item 如何随机选取字符串中的位置?
\end{itemize}
}

\cankao{
\begin{itemize}
  \item Beginning Perl for Bioinformatics, James Tisdall, O'Reilly Media, 2001.
  \item Perl语言入门(第六版),Randal L. Schwartz, brian d foy \& Tom Phoenix著,盛春\ 译,东南大学出版社,2012。
  \item Mastering Perl for Bioinformatics, James Tisdall, O'Reilly Media, 2003.
  \item 维基百科等网络资源。
\end{itemize}
}

\firstTail

\newpage
\otherHeader

\begin{enumerate}
  \item 随机(10分钟)
    \begin{itemize}
      \item 设置随机数种子:\verb=srand(time | $$)=
      \item 随机选取数组元素:\verb|$verbs[rand @verbs]|
    \end{itemize}
  \item 实验操作(80分钟)
    \begin{enumerate}
      \item 使用随机数生成器生成随机的语句
\begin{verbatim}
#!/usr/bin/perl

use strict;
use warnings;

my $count;
my $input;
my $sentence;
my $story;

my @nouns = (
    'Dad',     'TV',    'Mom',        'Groucho',
    'Rebecca', 'Harpo', 'Robin Hood', 'Joe and Moe',
);
my @verbs = (
    'ran to', 'giggled with', 'jumped with',
    'put hot sauce into the orange juice of',
    'exploded', 'dissolved', 'sang stupid songs with',
);
my @prepositions = (
    'at the store', 'over the rainbow',
    'just for the fun of it', 'at the beach',
    'before dinner', 'in New York City',
    'in a dream', 'around the world',
);

srand( time | $$ );

do {
    $story = '';
    for ( $count = 0 ; $count < 6 ; $count++ ) {
        $sentence =
            $nouns[ int( rand( scalar @nouns ) ) ] . " "
          . $verbs[ int( rand( scalar @verbs ) ) ] . " "
          . $nouns[ int( rand( scalar @nouns ) ) ] . " "
          . $prepositions[ int( rand( scalar @prepositions ) ) ] . '. ';
        $story .= $sentence;
    }

    print "\n", $story, "\n";
    print "\nType \"quit\" to quit, or press Enter to continue: ";
    $input = <STDIN>;
} until ( $input =~ /^\s*q/i );

\end{verbatim}

\otherTail
\newpage
\otherHeader

      \item 模拟DNA突变
\begin{verbatim}
#!/usr/bin/perl

use strict;
use warnings;

my $DNA = 'AAAAAAAAAAAAAAAAAAAAAAAAAAAAAA';
my $i;
my $mutant;

srand( time | $$ );
$mutant = mutate($DNA);
print "\nMutate DNA\n\n";
print "\nHere is the original DNA:\n\n";
print "$DNA\n";
print "\nHere is the mutant DNA:\n\n";
print "$mutant\n";
print "\nHere are 10 more successive mutations:\n\n";
for ( $i = 0 ; $i < 10 ; ++$i ) {
    $mutant = mutate($mutant);
    print "$mutant\n";
}

exit;

sub mutate {
    my ($dna) = @_;
    my (@nucleotides) = ( 'A', 'C', 'G', 'T' );
    my ($position) = randomposition($dna);
    my ($newbase) = randomnucleotide(@nucleotides);
    substr( $dna, $position, 1, $newbase );
    return $dna;
}

sub randomelement {
    my (@array) = @_;
    return $array[ rand @array ];
}

sub randomnucleotide {
    my (@nucleotides) = ( 'A', 'C', 'G', 'T' );
    return randomelement(@nucleotides);
}

sub randomposition {
    my ($string) = @_;
    return int rand length $string;
}
\end{verbatim}
    \end{enumerate}
\end{enumerate}

\otherTail

\end{document}
