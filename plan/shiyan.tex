\documentclass{TIJMUjiaoanSY}
\pagestyle{empty}

\begin{document}

\kecheng{分子生物计算}
\shiyan{实验 \ }
\jiaoshi{伊现富}
\zhicheng{讲师}
\riqi{2015年月日10:00-12:00}
\duixiang{生物医学工程与技术学院2013级生信班(本)}
\renshu{28}
\leixing{验证型}
\fenzu{一人一机}
\xueshi{2}
\jiaocai{Perl语言在生物信息学中的应用——基础篇}

\firstHeader
\maketitle
\thispagestyle{empty}

\mudi{
\begin{itemize}
  \item 了解。
  \item 熟悉。
  \item 掌握。
\end{itemize}
}

\fenpei{
\begin{itemize}
  \item 
  \item 
  \item 
  \item ()实验操作:
\end{itemize}
}

\cailiao{
\begin{itemize}
  \item 主要仪器:一台安装有Perl语言(Linux操作系统)的计算机。
\end{itemize}
}
\zhongdian{
\begin{itemize}
  \item 重点难点:。
  \item 解决策略:通过演示进行学习,通过练习熟练掌握。
\end{itemize}
}

\sikao{
\begin{itemize}
  \item 
  \item 
  \item 
\end{itemize}
}

\cankao{
\begin{itemize}
  \item Beginning Perl for Bioinformatics, James Tisdall, O'Reilly Media, 2001.
  \item Perl语言入门(第六版),Randal L. Schwartz, brian d foy \& Tom Phoenix著,盛春\ 译,东南大学出版社,2012。
  \item Mastering Perl for Bioinformatics, James Tisdall, O'Reilly Media, 2003.
  \item 维基百科等网络资源。
\end{itemize}
}

\firstTail

\newpage
\otherHeader

\begin{enumerate}
  \item 
  \item 
  \item 

\otherTail
\newpage
\otherHeader

  \item 
  \item 实验操作(分钟)
\end{enumerate}

\otherTail

%\parpic[fr]{\includegraphics[width=\textwidth]{}}

\end{document}
