\input{snippet/beamer_head.tex}
\begin{document}

%\includeonlyframes{current}

\logo{\includegraphics[height=0.08\textwidth]{tijmu.png}}

% 在每个Section前都会加入的Frame
\AtBeginSection[]
{
  \begin{frame}<beamer>
    %\frametitle{Outline}
    \frametitle{教学提纲}
    \setcounter{tocdepth}{3}
    \begin{multicols}{2}
      \tableofcontents[currentsection,currentsubsection]
      %\tableofcontents[currentsection]
    \end{multicols}
  \end{frame}
}
% 在每个Subsection前都会加入的Frame
\AtBeginSubsection[]
{
  \begin{frame}<beamer>
%%\begin{frame}<handout:0>
%% handout:0 表示只在手稿中出现
    \frametitle{教学提纲}
    \setcounter{tocdepth}{3}
    \begin{multicols}{2}
    \tableofcontents[currentsection,currentsubsection]
    \end{multicols}
%% 显示在目录中加亮的当前章节
  \end{frame}
}

% 为当前幻灯片设置背景
%{
%\usebackgroundtemplate{
%\vbox to \paperheight{\vfil\hbox to
%\paperwidth{\hfil\includegraphics[width=2in]{tijmu_charcoal.png}\hfil}\vfil}
%}
\begin{frame}[plain]
  \begin{center}
    {\Huge 分子生物计算\\}
    {\huge \textit{(Perl语言编程)}\\}
    \vspace{1cm}
    {\LARGE 天津医科大学\\}
    %\vspace{0.2cm}
    {\LARGE 生物医学工程与技术学院\\}
    \vspace{1cm}
    {\large 2018-2019学年上学期(秋)\\ 2016级生信班}
  \end{center}
\end{frame}
%}



\title[TIMTOWTDI]{TIMTOWTDI\\(条条大路通罗马)}
\author[Yixf]{伊现富(Yi Xianfu)}
\institute[TIJMU]{天津医科大学(TIJMU)\\ 生物医学工程与技术学院}
\date{2019年10月}

%\input{snippet/beamer_toc.tex}
\begin{frame}
  \titlepage
\end{frame}

\begin{frame}[fragile]
  \frametitle{帮助文档}
\begin{lstlisting}
# The Perl 5 language interpreter
man perl
# Look up Perl documentation in Pod format
man perldoc
perldoc perldoc

# Perl built-in function
perldoc -f BuiltinFunction
perldoc -f substr
# FAQ
perldoc -q FAQKeyword
perldoc -q random
# Perl predefined variable
perldoc -v PerlVariable
perldoc -v '$_'
\end{lstlisting}
\end{frame}

\begin{frame}[fragile]
  \frametitle{启用警告}
\begin{lstlisting}
# Method1
# 使用命令行选项-w
perl -w script.pl

# Method2
# 在命令解释行使用-w
#!/usr/bin/perl -w

# Method3
# 使用use
use warnings;
\end{lstlisting}
\end{frame}

\begin{frame}[fragile]
  \frametitle{字符串拼接}
\begin{lstlisting}
# Method1
$DNA3 = "$DNA1$DNA2"; print "$DNA3\n";

# Method2
print "$DNA1$DNA2\n";

# Method3
print $DNA1, $DNA2, "\n";

# Method4
$DNA3 = $DNA1 . $DNA2; print "$DNA3\n";

# Method5
$DNA3 = join "", $DNA1, $DNA2; print "$DNA3\n";
\end{lstlisting}
\end{frame}

\begin{frame}[fragile]
  \frametitle{文件读取}
\begin{lstlisting}
# Method1(只读取一行)
$protein = <PROTEINFILE>;

# Method2(读取所有行,一次性保存进数组)
@proteins = <PROTEINFILE>;

# Method3(依次读取每一行)
while (<PROTEINFILE>) {
  ...actions...
}

# Method4(读取所有行,一次性保存进标量)
$protein = do { local $/; <PROTEINFILE>; };
\end{lstlisting}
\end{frame}

\begin{frame}[fragile]
  \frametitle{获取数组元素个数}
\begin{lstlisting}
# Methods1
$num = scalar @bases;

# Methods2
$num = @bases;

#Methods3
$num = $#bases + 1;
\end{lstlisting}
\end{frame}

\begin{frame}[fragile]
  \frametitle{获取数组的第一个元素}
\begin{lstlisting}
# Method1
$first_gene = $genes[0];
#($first_gene) = $genes[0];

# Method2
($first_gene) = @genes;

# Method3
$first_gene = shift @genes;
# 注意:shift会影响原始的@genes数组
\end{lstlisting}
\end{frame}

\begin{frame}[fragile]
  \frametitle{获取数组的最后一个元素}
\begin{lstlisting}
# Method1
$last_gene = $genes[-1];

# Method2
$last_gene = $genes[$#genes];

# Method3
$last_gene = pop @genes;
# 注意:pop会影响原始的@genes数组
\end{lstlisting}
\end{frame}

\begin{frame}[fragile]
  \frametitle{变量值+1}
\begin{lstlisting}
# Methods1
$count++;

# Methods2
++$count;

# Methods3
$count += 1;

# Methods4
$count = $count + 1;
\end{lstlisting}
\end{frame}

\begin{frame}[fragile]
  \frametitle{计数核苷酸}
\begin{lstlisting}
# Methods1(使用数组)
@DNA = split( '', $DNA );
foreach $base (@DNA) { ...count... }
# Methods2(操作字符串)
for ( $pos=0 ; $pos < length $DNA ; ++$pos ) {
  $base = substr( $DNA, $pos, 1 ); ..count..
}
# Methods3(模式匹配)
while ( $DNA =~ /a/ig ) { $a++ }
while ( $DNA =~ /[^acgt]/ig ) { $e++ }
# Methods4(使用tr函数)
$a = ($DNA =~ tr/Aa/Aa/);
$basecount = ($DNA =~ tr/ACGTacgt/ACGTacgt/);
$nonbase = (length $DNA) - $basecount;
\end{lstlisting}
\end{frame}

\begin{frame}[plain]
\end{frame}

\begin{frame}
  \frametitle{考试}
  \begin{block}{考试安排}
    \begin{itemize}
      \item 时间:1.15(周二),10:30-12:10
      \item 地点:西楼510
      \item 时长:100分钟
      \item 总分:100分
    \end{itemize}
  \end{block}
  \pause
  \begin{block}{题型与分值}
    \begin{enumerate}
      \item 单选题:$25 \times 1  = 25$
      \item 多选题:$20 \times 1  = 20$
      \item 简答题:$4  \times 5  = 20$
      \item 编程题:$5  \times 5  = 25$
      \item 应用题:$1  \times 10 = 10$
    \end{enumerate}
  \end{block}
\end{frame}

\begin{frame}
  \frametitle{知识点}
  \begin{itemize}
    \item 生物学基础:拉丁语,数据库,碱基/氨基酸代码,限制酶,点突变
    \item R:常用包
    \item Markdown:基本语法,格式转换
    \item Git:基本使用
    \item Perl基础:CPAN,变量,赋值,use *,三大数据类型,上下文,子程序,调试器,排序,命令行参数,随机,关系数据库,my,编程流程,特殊变量,编程策略,构思步骤,字符串拼接,文件读取写入,perldoc,模块,设计理念,流程控制
    \item 数组:索引,操作,split,join
    \item 测试:条件操作符,字符串/数字/文件测试,真假
    \item 正则表达式:元字符,修饰符,匹配,替换,tr
    \item 编程:DNA突变,相似性计算,限制酶的正则表征,子程序传递,序列两两比较,阅读框翻译,……
  \end{itemize}
\end{frame}

\begin{frame}
  \frametitle{FATSQ}
  \begin{figure}
    \centering
    \includegraphics[width=0.35\textwidth]{fastq_01.png}
    \includegraphics[width=0.6\textwidth]{fastq_04.png}\\
    \vspace{1em}
    \includegraphics[width=0.35\textwidth]{fastq_03.jpg}
    \includegraphics[width=0.6\textwidth]{fastq_02.png}\\
    \vspace{1em}
    \includegraphics[width=0.9\textwidth]{fastq_05.png}
  \end{figure}
\end{frame}

\input{snippet/class_tail.tex}
