\input{snippet/beamer_head.tex}
\begin{document}

%\includeonlyframes{current}

\logo{\includegraphics[height=0.08\textwidth]{tijmu.png}}

% 在每个Section前都会加入的Frame
\AtBeginSection[]
{
  \begin{frame}<beamer>
    %\frametitle{Outline}
    \frametitle{教学提纲}
    \setcounter{tocdepth}{3}
    \begin{multicols}{2}
      \tableofcontents[currentsection,currentsubsection]
      %\tableofcontents[currentsection]
    \end{multicols}
  \end{frame}
}
% 在每个Subsection前都会加入的Frame
\AtBeginSubsection[]
{
  \begin{frame}<beamer>
%%\begin{frame}<handout:0>
%% handout:0 表示只在手稿中出现
    \frametitle{教学提纲}
    \setcounter{tocdepth}{3}
    \begin{multicols}{2}
    \tableofcontents[currentsection,currentsubsection]
    \end{multicols}
%% 显示在目录中加亮的当前章节
  \end{frame}
}

% 为当前幻灯片设置背景
%{
%\usebackgroundtemplate{
%\vbox to \paperheight{\vfil\hbox to
%\paperwidth{\hfil\includegraphics[width=2in]{tijmu_charcoal.png}\hfil}\vfil}
%}
\begin{frame}[plain]
  \begin{center}
    {\Huge 分子生物计算\\}
    {\huge \textit{(Perl语言编程)}\\}
    \vspace{1cm}
    {\LARGE 天津医科大学\\}
    %\vspace{0.2cm}
    {\LARGE 生物医学工程与技术学院\\}
    \vspace{1cm}
    {\large 2018-2019学年上学期(秋)\\ 2016级生信班}
  \end{center}
\end{frame}
%}



\title[其他]{第10..13章\quad GenBank、PDB、BLAST、其他}
\author[Yixf]{伊现富(Yi Xianfu)}
\institute[TIJMU]{天津医科大学(TIJMU)\\ 生物医学工程与技术学院}
\date{2016年12月}

\input{snippet/beamer_toc.tex}


\section{模式匹配}
\begin{frame}[fragile]
  \frametitle{模式匹配 | 界定符 | =$\sim$m}
\begin{lstlisting}
if ( $line =~ /^\/\/\n/ ) {
  last;
}

if ( $line =~ m!//\n! ) {
  last;
}
\end{lstlisting}
\end{frame}

\begin{frame}[fragile]
  \frametitle{模式匹配 | 修饰符 | /m}
\begin{lstlisting}
#!/usr/bin/perl

use warnings;

"AAC\nGTT" =~ /^.*$/;
print "Without /m:\n", $&, "\n";
#Without /m:
#Use of uninitialized value $& in print at XXX.pl line N.

"AAC\nGTT" =~ /^.*$/m;
print "With /m:\n", $&, "\n";
#With /m:
#AAC
\end{lstlisting}
\end{frame}

\begin{frame}[fragile]
  \frametitle{模式匹配 | 修饰符 | /s}
\begin{lstlisting}
#!/usr/bin/perl

use warnings;

"AAC\nGTT" =~ /^.*$/;
print "Without /s:\n", $&, "\n";
#Without /s:
#Use of uninitialized value $& in print at XXX.pl line N.

"AAC\nGTT" =~ /^.*$/s;
print "With /s:\n", $&, "\n";
#With /s:
#AAC
#GTT
\end{lstlisting}
\end{frame}

\begin{frame}[fragile]
  \frametitle{模式匹配 | 捕获}
\begin{lstlisting}
#!/usr/bin/perl

use strict; use warnings;

my $alphabet = join "", 'a' .. 'z';
$alphabet =~ /k(lmnop)q/;
print $1, "\n\n";
#lmnop

$alphabet =~ /(((a)b)c)/;
print "First pattern = ",  $1, "\n";
print "Second pattern = ", $2, "\n";
print "Third pattern = ",  $3, "\n";
#First pattern = abc
#Second pattern = ab
#Third pattern = a
\end{lstlisting}
\end{frame}

\section{输入记录分隔符}
\begin{frame}[fragile]
  \frametitle{输入记录分隔符}
\begin{lstlisting}
my $save_input_separator = $/;

$/ = "//\n";
$record = <GBFILE>;

$/ = $save_input_separator;
\end{lstlisting}  
\end{frame}

\section{读取文件}
\begin{frame}[fragile]
  \frametitle{读取文件 | tell \& seek}
\begin{lstlisting}
for (;;) {
    for ($curpos = tell(FILE); $_ = <FILE>; $curpos = tell(FILE)) {
        # search for some stuff and put it into files
    }
    sleep($for_a_while);
    seek(FILE, $curpos, 0);
}
\end{lstlisting}
\end{frame}

\section{文件夹处理}
\begin{frame}[fragile]
  \frametitle{文件夹处理 | 递归}
\begin{lstlisting}[basicstyle=\scriptsize\tt,numberstyle=\tiny]
#!/usr/bin/perl
use strict; use warnings;
list_recursively('pdb');

sub list_recursively {
    my ($directory) = @_;
    my @files = ();
    unless ( opendir( DIRECTORY, $directory ) ) {
        print "Cannot open directory $directory!\n";
        exit;
    }
    @files = grep ( !/^\.\.?$/, readdir(DIRECTORY) );
    closedir(DIRECTORY);
    foreach my $file (@files) {
        if ( -f "$directory/$file" ) {
            print "$directory/$file\n";
        }
        elsif ( -d "$directory/$file" ) {
            list_recursively("$directory/$file");
        }
    }
}
\end{lstlisting}
\end{frame}

\begin{frame}[fragile]
  \frametitle{文件夹处理 | 模块}
\begin{lstlisting}
#!/usr/bin/perl

use strict;
use warnings;
use File::Find;
#perldoc File::Find

find( \&my_sub, ('pdb') );

sub my_sub {
    -f and ( print $File::Find::name, "\n" );
}
\end{lstlisting}
\end{frame}

\section{格式化输出}
\begin{frame}[fragile]
  \frametitle{格式化输出 | printf}
\begin{lstlisting}
while(<>) {
  /^ATOM/ or next;

  my($n, $x, $y, $z, $element)
    = ($_ =~ /^.{6}(.{5}).{19}(.{8})(.{8})(.{8}).{22}(..)/);

  $n      =~ s/^\s*//;
  $element =~ s/^\s*//;

  if (($n == 1) or ($n == 1078)) {
    printf "%8.3f%8.3f%8.3f %2s\n", $x, $y, $z, $element;
  }
}
\end{lstlisting}
\end{frame}

\begin{frame}[fragile]
  \frametitle{格式化输出 | printf}
\begin{lstlisting}
my $first  = '3.14159265';
my $second  = 76;
my $third = "Hello world!";

printf STDOUT "A float: %6.4f An integer: %-5d and a string: %s\n", $first, $second,  $third;
#A float:  3.1416 An integer: 76    and a string: Hello world!
\end{lstlisting}
\end{frame}

\begin{frame}[fragile]
  \frametitle{格式化输出 | here文档}
\begin{lstlisting}
#!/usr/bin/perl
use strict; use warnings;
my $DNA = 'AAACCCCCCGGGGGGGGTTTTTT';
for ( my $i = 0 ; $i < 2 ; ++$i ) {
    print <<HEREDOC;
     On iteration $i of the loop!
    $DNA

HEREDOC
}
#     On iteration 0 of the loop!
#    AAACCCCCCGGGGGGGGTTTTTT
#
#     On iteration 1 of the loop!
#    AAACCCCCCGGGGGGGGTTTTTT
#
\end{lstlisting}
\end{frame}

\begin{frame}[fragile]
  \frametitle{格式化输出 | format \& write}
\begin{lstlisting}[basicstyle=\footnotesize\tt,numberstyle=\scriptsize]
#!/usr/bin/perl
use strict; use warnings;
my $id          = 'A0000';
my $description = 'Highly weird DNA.  This DNA is so unlikely!';
my $DNA = 'AAAAAACCCCCCCCCCCCCCGGGGGGGGGGGGGGGGGGGGGGTTTTTTTTTTTTTTTTTTTTT';
# Define the format
format STDOUT =
# The header line
>@<<<<<<<<< @<<<<<<<<<<<<<<<<<<<<<<<<<<<<<<<<<<<...
$id,        $description
# The DNA lines
^<<<<<<<<<<<<<<<<<<<<<<<<<<<<<<<<<<<<<<<<<<<<<<<<<~~
$DNA
.
# Print the fasta-formatted DNA output
write;
\end{lstlisting}
\end{frame}

\begin{frame}[fragile]
  \frametitle{格式化输出 | format \& write}
\begin{lstlisting}[basicstyle=\footnotesize\tt,numberstyle=\scriptsize]
>A0000      Highly weird DNA.  This DNA is so un...
AAAAAACCCCCCCCCCCCCCGGGGGGGGGGGGGGGGGGGGGGTTTTTTTT
TTTTTTTTTTTTT
\end{lstlisting}
\end{frame}

\section{运行外部程序}
\begin{frame}[fragile]
  \frametitle{运行外部程序}
\begin{lstlisting}[basicstyle=\small\tt]
my $filename = $ARGV[0];
my $stride = '/usr/local/bin/stride';
my $options = '';
# 捕获输出
my @results = `$stride $options $filename`;
my $now = `date`;
my @functions = qw{ int rand length };
my %about;
foreach (@functions) {
  #$about{$_} = `perldoc -t -f $_`;
  $about{$_} = qx(perldoc -t -f $_);
}

# 不捕获输出,返回值为程序退出状态
system "$stride $options $filename";
system 'date';
system 'tar', 'cvf', $tarfile, @dirs;
\end{lstlisting}
\end{frame}

\section{浮点数比较}
\begin{frame}[fragile]
  \frametitle{浮点数比较}
\begin{lstlisting}[basicstyle=\small\tt]
#!/usr/bin/perl

if ( 10 / 3 == ( ( 1 / 3 ) * 10 ) ) {
    print "Success!\n";
}
else { print "Failure!\n"; }
#Failure!

if ( abs( 10/3 - ( ( 1/3 ) * 10 ) ) < 1e-10 ) {
    print "Right!\n";
    print "E=", abs(10/3 - ( (1/3) * 10 )), "\n";
}
else { print "Wrong!\n"; }
#Right!
#E=4.44089209850063e-16
\end{lstlisting}
\end{frame}

\section{BioPerl}
\begin{frame}[fragile]
  \frametitle{BioPerl}
\begin{lstlisting}[basicstyle=\footnotesize\tt,numberstyle=\scriptsize]
use Bio::SeqIO;

my $string = ">SEQ1\nacgt\n>revseq1\ntgca "; 
my $format = "fasta";

my $stringfh = IO::String->new($string); 
open($stringfh, "<", $string) or die "Could not open string for reading: $!";

my $seqio = Bio::SeqIO-> new(-fh     => $stringfh,
                             -format => $format,
                            );

while( my $seq = $seqio->next_seq ) {
   print $seq->id . ' = ' . $seq->seq() . "\n"; 
}
\end{lstlisting}
\end{frame}

\begin{frame}[fragile]
  \frametitle{BioPerl}
\begin{lstlisting}[basicstyle=\footnotesize\tt,numberstyle=\scriptsize]
use strict; use Bio::SeqIO;
my $input_file = shift;
my $seq_in = Bio::SeqIO->new( -format => 'embl',
                              -file   => $input_file,
                             );
my @seq_array; 
while ( my $seq = $seq_in->next_seq() ) {
   push(@seq_array,$seq);
}
@seq_array = sort { $a->length <=> $b->length } @seq_array;
my $total = 0; 
my $count = 0; 
for my $seq ( @seq_array ) {
   $total += $seq->length;
   $count++;
}
print "Mean length ", $total/$count, " Median ", 
      $seq_array[$count/2]->length, "\n";
\end{lstlisting}
\end{frame}

\begin{frame}[fragile]
  \frametitle{BioPerl}
\begin{lstlisting}[basicstyle=\footnotesize\tt,numberstyle=\scriptsize]
use Bio::SeqIO;

my $usage = "x2y.pl infile informat outfile outformat "; 
my $infile = shift or die $usage; 
my $informat = shift or die $usage; 
my $outfile = shift or die $usage; 
my $outformat = shift or die $usage;

my $seq_in = Bio::SeqIO->new( -file   => "$infile",
                              -format => $informat,
                            );
my $seq_out = Bio::SeqIO->new( -file   => ">$outfile",
                               -format => $outformat,
                             );
while (my $inseq = $seq_in->next_seq) {
   $seq_out->write_seq($inseq);
}
\end{lstlisting}
\end{frame}

\begin{frame}[fragile]
  \frametitle{BioPerl}
\begin{lstlisting}[basicstyle=\tiny\tt,numberstyle=\tiny]
use Bio::SeqIO;
my $usage = "splitgb.pl infile "; my $infile = shift or die $usage;
my $inseq = Bio::SeqIO->new( -file   => "<$infile", -format => 'Genbank',);
my %outfiles = ( human => {
                         Genbank => Bio::SeqIO->new(
                                                    -file   => '>human.gb',
                                                    -format => 'Genbank',
                                                    ),
                         Fasta   => Bio::SeqIO->new(
                                                    -file   => '>human.fa',
                                                    -format => 'Fasta',
                                                    ),
                         },
                 other => {
                         Genbank => Bio::SeqIO->new(
                                                    -file   => '>other.gb',
                                                    -format => 'Genbank',),
                         Fasta   => Bio::SeqIO->new(
                                                    -file => '>other.fa',
                                                    -format => 'Fasta',),
                         }
               );
while (my $seqin = $inseq->next_seq) {
   if ($seqin->species->binomial =~ m/Homo sapiens/) {
       $outfiles{'human'}->{'Genbank'}->write_seq($seqin);
       $outfiles{'human'}->{'Fasta'}->write_seq($seqin);
   } else {
       $outfiles{'other'}->{'Genbank'}->write_seq($seqin);
       $outfiles{'other'}->{'Fasta'}->write_seq($seqin);
   }
}
\end{lstlisting}
\end{frame}

\begin{frame}[fragile]
  \frametitle{BioPerl}
\begin{lstlisting}[basicstyle=\footnotesize\tt,numberstyle=\scriptsize]
use Bio::DB::GenBank;
use Bio::DB::Query::GenBank;

$query = "Arabidopsis[ORGN] AND topoisomerase[TITL] and 0:3000[SLEN]"; 
$query_obj = Bio::DB::Query::GenBank->new(-db => 'nucleotide', 
                                          -query => $query );

$gb_obj = Bio::DB::GenBank->new;

$stream_obj = $gb_obj->get_Stream_by_query($query_obj);

while ($seq_obj = $stream_obj->next_seq) {
    # do something with the sequence object    
    print $seq_obj->display_id, "\t", $seq_obj->length, "\n";
}
\end{lstlisting}
\end{frame}


\input{snippet/class_tail.tex}
